\section{Glossary}
\begin{description}
	\item[CLI] Command line interface
	
	\item[Master-Worker] There is one Master scheduler which schedules the tasks. The other instances are the workers. Its the opposite of Mutiple-queue.

	\item[Nodes] The bwUniCluster is divided into 522 nodes which have 16 to 32 cores with 64 to 1 TB main memory and 332,8 to 614,4 GFLOPS for more information please visit \href{https://www.scc.kit.edu/dienste/bwUniCluster.php}{https://www.scc.kit.edu/dienste/bwUniCluster.php}

	\item[Multiple-queue]Each scheduler has its own queue. All schedulers are equal and use task-stealing to avoid idle. Its the opposite of Master-Worker.

	\item[GFLOPS] 1 GFLOPS means $10^9$ floating-point operations per second

	\item[Moab workload Manager] Moab provides numerous interfaces allowing it to monitor and manage most services and resources. It also possesses flexible interfaces to allow it to interact with peer services and applications as both a broker and an information service. This appendix is designed to provide a general overview and links to more detailed interface documentation

	\item [SimLap for Elemantary- and Astro-    Particle Physics (SCC)] Support scientific groups with concept „S.P.O.R.A.D.I.C. services for simulations (Standardisation, Parallelising, Optimization, Release, Adaptation, Data Intensive Computing) \href {https://www.scc.kit.edu/en/research/7047.php}{https://www.scc.kit.edu/en/research/7047.php}

	\item[MPI] The communication environment which is used by the scheduler to communicate with each other 

	\item[Data mining] The process to extract useful information from a big amount of data

	\item[Scheduling strategies] The program provides multiple strategies to schedule the tasks

	\item[Scientific code] Code which contains calculations for physical simulations
	
	\item[HPC] Shortcut for high-performance computer(supercomputer)
	
	\item[dummy code] Pre-defined module with an short scientific code example for a short computation to test multiple strategies
	
	\item[schedular] Program module which is managing the way and order of execution of many processes
	
	\item[task] The Scientific code contains independent tasks which can be calculated simultaneously
	
	\item[GUI] GUI stands for graphical user interface. A way the user can interact with the program graphical
	
	\item[shell script] Computer program designed to be run by the Unix shell, a command line interpreter
\end{description}