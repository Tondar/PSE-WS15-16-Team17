\section{Introduction}


%Die Anwendung wird in wissenschaftlichen Code eingebunden und verwaltet Ausführung einer gesamten Berechnung, die in einzelne tasks unterteilt ist. Tasks stellen hierbei einzelne Simulationsinstanzen mit spezifischen Parametern dar. Durch die Ausführung der tasks werden Metainformationen gespeichert, mit dem Ziel einen möglichst optimalen Programmfluss zu gewährleisten. Die gespeicherten Metainformationen werden von einem Datamining Modul ausgewertet. Daraus kann dynamisch zur Laufzeit der Berechnung die Laufzeit optimiert werden. Die aufgezeichneten Informationen können anschließend wahlweise grafisch ausgegeben werden.


%The application is meant to be linked into scientific code, functioning as runtime managing service for scientific simulations constisting of huge ammounts of tasks beeing single calculations unter specified changing parameters.
%Gathering meta informations about single calculation runs the schedular is able to use these to boost performance and throughtput of the whole scientific simulation. If chosen, the all data gathered can be plotted and put out graphically.


The application works as a runtime administration service for scientific simulations. Linked into scientific code, it manages huge computations consisting of large amounts of individual tasks. In this context tasks are specified as particular calculation runs with a given set of parameters. The scheduling module will load prestored data from the database module and use it as initial information for scheduling. Gathering meta information about separately performed calculations, the scheduler is able to dynamically adjust administration in order to increase the performance of the whole computation. Obtained data can be plotted after the simulation is finished, if the user chooses to do so. In any case, it will be stored on a database module to be used as initial information for scheduling calculations in future simulations. 