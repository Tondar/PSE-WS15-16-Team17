{
\newcounter{funcNFR}
\setcounter{funcNFR}{10}
\renewcommand{\labelitemi}{
	\ifnum \value{funcNFR}<10$/NF 0\arabic{funcNFR} /$\addtocounter{funcNFR}{10}
	\else $/NF \arabic{funcNFR} /$\addtocounter{funcNFR}{10}\fi
}

\section{Non-functional requirements}
	\begin{itemize}
		\item The dynamic scheduler has to be parallelized and uses the Massage Passing Interface (MPI) for the communication between different CPU's
		\item The program must run in a 'MPI World' up to 3.000.000 CPU's
		\item The dynamic scheduler must have a minimum of overhead. The focus should be on running the scientific simulation on the CPU and not the scheduler
		\item It must be possible to connect the scientific code with the scheduler with basic C++ skills
		\item The command line interface of the dynamic scheduler has to be intuitive
		\item The scheduler must not manipulate scientific tasks
		\item The scheduler must not manipulate scientific code
		\item The scheduler must not affect the result of the scientific simulations
		\item It must be possible to integrate new scheduling strategies easily
		\item It must be easy to continue the development of the dynamic scheduler
	\end{itemize}
}