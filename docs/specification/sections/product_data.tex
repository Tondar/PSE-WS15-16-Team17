{


\newcounter{funcD}
\setcounter{funcD}{10}
\renewcommand{\labelitemi}{
	\ifnum \value{funcD}<10$/D 0\arabic{funcD} /$\addtocounter{funcD}{10}
	\else $/D \arabic{funcD} /$\addtocounter{funcD}{10}\fi
	}
\section{Product data}

	\subsection{Scheduler}
		\begin{itemize}
			\item All tasks of the computation, where a task consists of:
				\begin{itemize}
					\item \textbf{Data\_Type} (e.g. structure or array);
					\item \textbf{Data} of this type
				\end{itemize}
			\item Command line parameter that chooses scheduling strategy
			\item Run arguments for the scientific code initialization
		\end{itemize}


	\subsection{Bookkeeping}
		\begin{itemize}
			\item Hosting process for each task
			\item Timestamp when a task appears
			\item Timestamp when a task is started
			\item Timestamp when a task is finished
			\item Task parameters when a task appears
			\item Task parameters when a task is started
			\item Task parameters when a task is finished
			\item Parent process of the appearing task
			\item Hosting process of the starting task
			\item Hosting process of the finishing task
			\item Timestamp when a intercommunication between processes started
			\item Timestamp when a intercommunication between processes finished
			\item Which CPUs (processors' ranks) communicated and the direction of the communication (e.g. from p1 to p2) 
		\end{itemize}
			
			
	\subsection{Statistics}
		\begin{itemize}
			\item The runtime for each task depending on task parameters (Which data are important for the statistics is defined in the task data type)
		\end{itemize}
}