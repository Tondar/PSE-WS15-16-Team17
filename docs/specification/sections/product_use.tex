\section{Product use}

\subsection{Scope of application}
The dynamic scheduler is for scientists and researchers at 'SimLab for Elementary- and Astro- Particle' (SCC). Scientists and researchers use the dynamic scheduler to run scientific code with different scheduling algorithms. Furthermore the dynamic scheduler collects statistics, does bookkeeping and visualizes the statistics.


\subsection{Target group}
%be more general
Target groups are scientists and researchers at 'SimLab for Elementary- and Astro- Particle' (SCC). The dynamic scheduler is called via the command line. Users need to have basic skills of using the command line and working on the high performance computers (HPC) with 'Moab Workload Manager'. Moreover scientists and researchers need to have basic skills in programming and compiling c++ with Message Passing Interface (MPI) to link scientific code into the dynamic scheduler.
(Optional: The dynamic scheduler provides a graphical user interface (GUI) allowing simulation runs on the HPC)


\subsection{Operation conditions}

The following requirements have to be met:
\begin{itemize}
	\item Parallel computer environment (HPC)
	\item Sufficient time table for finishing the job on the HPC
	\item Installed Message Passing Interface (MPI)
	\item Scientific code providing the following interface:
		\begin{itemize}
			\item code\_preprocessing\_master(run arguments): set of initial tasks
			\item code\_preprocessing\_slave(run arguments): void
			\item  code\_postprocessing\_master(void): void
			\item  code\_postprocessing\_slave(void): void
		\end{itemize}
	\item Scientific code using the following interface:
		\begin{itemize}
			\item run\_task(task): run result
		\end{itemize}
\end{itemize}